\documentclass[11pt]{article}

% arXiv recommended packages
\usepackage[utf8]{inputenc}
\usepackage[T1]{fontenc}
\usepackage{hyperref}
\usepackage{url}
\usepackage{booktabs}
\usepackage{amsfonts}
\usepackage{amsmath}
\usepackage{amssymb}
\usepackage{graphicx}
\usepackage{xcolor}
\usepackage{listings}
\usepackage{algorithm}
\usepackage{algorithmic}
\usepackage[margin=1in]{geometry}

% Code listing style
\lstset{
  basicstyle=\ttfamily\small,
  breaklines=true,
  frame=single,
  language=Python,
  showstringspaces=false,
  keywordstyle=\color{blue},
  commentstyle=\color{gray},
  stringstyle=\color{orange}
}

\title{4D Persona Architecture: A Dimensional Model for Embodied AI Agents}

\author{
  Eyal Nof\\
  \texttt{[eyalnof123@gmail.com]}
}

\date{December 28, 2025}

\begin{document}

\maketitle

\begin{abstract}
We introduce the \textbf{4D Persona Architecture}, a novel framework for AI persona systems where agent identity is computed as a dynamic position in a four-dimensional space rather than declared through static text descriptions. The four dimensions are: \textbf{Emotional} (affective state derived from real-world data), \textbf{Relational} (position in a knowledge graph of relationships), \textbf{Linguistic} (voice, dialect, and vocabulary identity), and \textbf{Temporal} (evolution trajectory through time). Unlike traditional approaches where personas are ``costumes'' the model wears, our architecture treats persona as a coordinate that moves through dimensional space based on ground truth data, contextual triggers, and temporal dynamics. We demonstrate this architecture through a reference implementation (Soccer-AI) and show that computed personas exhibit qualitatively different behaviors than declared personas. This work introduces the concept of \textbf{Embodied RAG}, where Retrieval-Augmented Generation is extended to include persona computation from reality, not just fact retrieval.
\end{abstract}

\section{Introduction}

Current AI systems define personas through static text descriptions in system prompts:

\begin{quote}
``You are a helpful assistant who is enthusiastic about football.''
\end{quote}

This approach has fundamental limitations. The persona never changes regardless of context. Emotions are performed rather than derived from reality. There is no dimensional variation in response to stimuli, and no memory of persona evolution over time.

We observe that static personas create \textit{actors}, not \textit{characters}. The AI says the right words but does not \textit{inhabit} the identity. This paper introduces a paradigm shift: instead of describing \textbf{who} the persona is, we define \textbf{where} the persona is in a four-dimensional space.

Our contributions are:
\begin{enumerate}
    \item \textbf{4D Persona Model}: A formal definition of persona as a coordinate $P(t) = (x, y, z, t)$ in dimensional space
    \item \textbf{Embodied RAG}: An extension of RAG where persona state is computed from ground truth data
    \item \textbf{Temporal Dynamics}: Introduction of trajectory, velocity, and momentum for persona evolution
    \item \textbf{Reference Implementation}: Soccer-AI, demonstrating the architecture with Premier League fan personas
\end{enumerate}

\section{Related Work}

\subsection{Retrieval-Augmented Generation}
RAG \cite{lewis2020retrieval} revolutionized how AI systems access knowledge by retrieving relevant information at query time rather than relying solely on parametric knowledge. However, RAG focuses on \textit{fact retrieval}, leaving persona definition untouched. The system prompt remains static regardless of retrieved content.

\subsection{Persona-Based Dialogue Systems}
Prior work on persona-based dialogue \cite{zhang2018personalizing} introduced persona-conditioned response generation. However, these approaches use static persona descriptions (e.g., ``I have a dog'', ``I work as a teacher'') that do not change based on external events or temporal dynamics.

\subsection{Emotional AI}
Affective computing \cite{picard1997affective} has explored emotional expression in AI systems. Most approaches either classify user emotion or generate predefined emotional responses. Our work differs by \textit{deriving} emotional state from external ground truth data.

\subsection{Knowledge Graphs in Dialogue}
Knowledge-grounded dialogue \cite{dinan2018wizard} uses structured knowledge to inform responses. We extend this by positioning the persona \textit{within} the knowledge graph, enabling relationship-aware behavior.

\section{The 4D Persona Architecture}

\subsection{Core Formulation}

We define a persona state $P$ in a four-dimensional space:

\begin{equation}
P(t) = (x, y, z, t)
\end{equation}

Where:
\begin{itemize}
    \item $x \in [-1, 1]$: Emotional dimension (negative to positive affect)
    \item $y \in \mathcal{G}$: Relational dimension (position in knowledge graph $\mathcal{G}$)
    \item $z \in \mathcal{L}$: Linguistic dimension (dialect/vocabulary space $\mathcal{L}$)
    \item $t \in \mathbb{N} \times \mathcal{H}$: Temporal dimension (time step and history $\mathcal{H}$)
\end{itemize}

The full state space is the Cartesian product:
\begin{equation}
\mathcal{P} = E \times R \times L \times T
\end{equation}

\subsection{Dimension X: Emotional}

The emotional dimension represents the affective state of the persona, \textbf{derived from real-world ground truth data} rather than declared.

\begin{equation}
x = f_E(D) = \frac{2 \cdot \text{points}(D)}{\text{max\_points}(D)} - 1
\end{equation}

Where $D$ is ground truth data (e.g., recent match results for a sports fan persona). This ensures:
\begin{itemize}
    \item Emotion changes automatically when reality changes
    \item Emotion has \textit{provenance} (can explain why)
    \item Emotion exists on a continuum, not discrete categories
\end{itemize}

\subsection{Dimension Y: Relational}

The relational dimension positions the persona within a knowledge graph $\mathcal{G} = (V, E, W)$ with vertices $V$, edges $E$, and weights $W$.

\begin{equation}
y = f_R(\mathcal{G}, c) = \max_{e \in E_{activated}(c)} W(e)
\end{equation}

Where $E_{activated}(c)$ are edges activated by context $c$. This enables:
\begin{itemize}
    \item Multi-hop reasoning through graph traversal
    \item Relationship intensity based on edge weights
    \item Context-dependent relationship activation
\end{itemize}

\subsection{Dimension Z: Linguistic}

The linguistic dimension captures voice, dialect, and vocabulary identity:

\begin{equation}
z = f_L(entity) = (\delta, \mathcal{V}, \mathcal{C})
\end{equation}

Where $\delta$ is dialect distinctiveness, $\mathcal{V}$ is vocabulary mapping, and $\mathcal{C}$ is constraints (forbidden terms). This affects \textit{how} things are said, not \textit{what} is said.

\subsection{Dimension T: Temporal}

The temporal dimension introduces trajectory dynamics:

\begin{equation}
\tau(t, w) = [P(t-w), P(t-w+1), \ldots, P(t)]
\end{equation}

With velocity and momentum:
\begin{align}
v(t) &= P(t) - P(t-1) \\
\hat{P}(t+1) &= P(t) + v(t)
\end{align}

This enables prediction of future persona states and ensures temporal continuity.

\subsection{State Transition Function}

The persona evolves according to:

\begin{equation}
P(t+1) = f(P(t), c(t), D(t))
\end{equation}

Where $c(t)$ is context and $D(t)$ is ground truth data. The transition decomposes:

\begin{equation}
f(P, c, D) = (f_E(D), f_R(\mathcal{G}, c), f_L(entity), f_T(P))
\end{equation}

\section{Embodied RAG}

We introduce \textbf{Embodied RAG} as an extension of traditional RAG:

\begin{table}[h]
\centering
\begin{tabular}{lcc}
\toprule
\textbf{Aspect} & \textbf{Traditional RAG} & \textbf{Embodied RAG} \\
\midrule
Persona & Static text & Computed 4D coordinate \\
Emotion & Declared & Derived from data \\
Retrieval & Facts only & Facts + relationships \\
Memory & Context window & Full trajectory \\
\bottomrule
\end{tabular}
\caption{Comparison of Traditional RAG and Embodied RAG}
\end{table}

The key insight is that \textbf{RAG retrieves facts; Embodied RAG computes feelings}.

Traditional RAG:
\begin{equation}
\text{Response} = \text{LLM}(\text{static\_persona} + \text{retrieved\_chunks} + \text{query})
\end{equation}

Embodied RAG:
\begin{equation}
\text{Response} = \text{LLM}(\text{compute\_persona}(D, \mathcal{G}, t) + \text{retrieved\_chunks} + \text{query})
\end{equation}

\section{Dynamic System Prompt Synthesis}

The 4D position is synthesized into a dynamic system prompt per-request:

\begin{algorithm}
\caption{System Prompt Synthesis}
\begin{algorithmic}[1]
\STATE \textbf{Input:} Persona state $P(t) = (x, y, z, t)$, base prompt $B$
\STATE \textbf{Output:} Synthesized system prompt $S$
\STATE $S \leftarrow B$
\STATE $S \leftarrow S + $ EmotionalInjection$(x)$
\IF{$y.\text{activated}$}
    \STATE $S \leftarrow S + $ RelationalInjection$(y)$
\ENDIF
\IF{$z.\text{dialect} \neq \text{neutral}$}
    \STATE $S \leftarrow S + $ LinguisticInjection$(z)$
\ENDIF
\IF{$t.\text{memory} \neq \emptyset$}
    \STATE $S \leftarrow S + $ TemporalInjection$(t)$
\ENDIF
\RETURN $S$
\end{algorithmic}
\end{algorithm}

Every request receives a persona calibrated to current reality.

\section{Reference Implementation: Soccer-AI}

We demonstrate the 4D Persona Architecture through Soccer-AI, a conversational system where users interact with AI ``fans'' of Premier League football clubs.

\subsection{Emotional Computation}
Mood is derived from actual match results stored in a database:
\begin{lstlisting}
def calculate_mood(club, num_matches=5):
    matches = db.query(
        "SELECT result FROM matches
         WHERE team = ?
         ORDER BY date DESC LIMIT ?",
        (club, num_matches)
    )
    points = sum(3 if m=='W' else 1 if m=='D' else 0
                 for m in matches)
    form = points / (num_matches * 3)
    return derive_mood(form)
\end{lstlisting}

\subsection{Relational Activation}
Rivalry detection triggers relationship-specific behavior:
\begin{lstlisting}
RIVALRIES = {
    "arsenal": ["tottenham", "chelsea"],
    "liverpool": ["manchester_united", "everton"],
    # ...
}

def check_rivalry(club, message):
    rivals = RIVALRIES.get(club, [])
    for rival in rivals:
        if rival in message.lower():
            return activate_rivalry(club, rival)
\end{lstlisting}

\subsection{Linguistic Identity}
Regional dialects applied based on club:
\begin{lstlisting}
DIALECTS = {
    "liverpool": {"phrases": ["la", "sound", "boss"],
                  "style": "Scouse"},
    "newcastle": {"phrases": ["howay", "canny"],
                  "style": "Geordie"},
}
\end{lstlisting}

\subsection{Qualitative Comparison}

\textbf{Traditional approach} (static persona):
\begin{quote}
User: ``What do you think about Tottenham?''

Response: ``As an Arsenal fan, I don't like Tottenham. They are our rivals. The North London Derby is always intense.''
\end{quote}

\textbf{4D Persona approach} (computed state: mood=anxious, rivalry=activated):
\begin{quote}
User: ``What do you think about Tottenham?''

Response: ``Spurs? \textit{sighs} Look, normally I'd be all over them - what do we think of Tottenham and all that - but after that Chelsea result, I'm just... we need to sort ourselves out before the derby, you know? Mind you, they're not exactly setting the world alight either, are they?''
\end{quote}

The second response weaves together current emotional state (anxious from recent results), rivalry activation, and authentic voice.

\section{Discussion}

\subsection{Grounded vs. Declared Emotion}
The critical distinction is between emotion that is \textit{told} versus emotion that is \textit{derived}. When an AI is told ``be enthusiastic,'' it performs enthusiasm. When enthusiasm emerges from winning streak data, it \textit{has} enthusiasm.

\subsection{Persona as Trajectory}
Traditional personas are points. Our architecture treats persona as a \textit{path} through 4D space. This enables questions like ``Where is this persona heading?'' and ``How stable is this character?''

\subsection{The Costume Problem Solved}
Static personas are costumes. The 4D architecture creates characters that inhabit their identity because that identity is computed from the same reality they discuss.

\subsection{Limitations}
\begin{itemize}
    \item Requires ground truth data sources
    \item Knowledge graph construction is domain-specific
    \item Temporal persistence requires state management
\end{itemize}

\section{Applications Beyond Sports}

The architecture generalizes to any domain where personas should be grounded in reality:

\begin{itemize}
    \item \textbf{Customer Service}: Agent patience derived from actual wait time metrics
    \item \textbf{Education}: Tutor encouragement calibrated to real student progress
    \item \textbf{Healthcare}: Wellness companion tone based on actual health data
    \item \textbf{Gaming}: NPC emotions grounded in actual game world state
\end{itemize}

\section{Conclusion}

We presented the 4D Persona Architecture, a paradigm shift from describing AI personas to computing them. By treating persona as a position in four-dimensional space (Emotional, Relational, Linguistic, Temporal), we enable:

\begin{enumerate}
    \item \textbf{Grounded emotion}: Feelings derived from real data
    \item \textbf{Relationship awareness}: Graph-based reasoning
    \item \textbf{Linguistic identity}: Consistent voice
    \item \textbf{Temporal continuity}: Memory and trajectory
\end{enumerate}

The AI does not play a character. It lives one.

This work opens new directions for persona design in AI systems, moving from static descriptions to dynamic, reality-grounded, temporally-evolving identities.

\section*{Acknowledgments}
[To be added]

\bibliographystyle{plain}
\begin{thebibliography}{9}

\bibitem{lewis2020retrieval}
Lewis, P., et al. (2020).
Retrieval-Augmented Generation for Knowledge-Intensive NLP Tasks.
\textit{NeurIPS 2020}.

\bibitem{zhang2018personalizing}
Zhang, S., et al. (2018).
Personalizing Dialogue Agents: I have a dog, do you have pets too?
\textit{ACL 2018}.

\bibitem{picard1997affective}
Picard, R. W. (1997).
\textit{Affective Computing}.
MIT Press.

\bibitem{dinan2018wizard}
Dinan, E., et al. (2018).
Wizard of Wikipedia: Knowledge-Powered Conversational Agents.
\textit{ICLR 2019}.

\end{thebibliography}

\appendix

\section{Implementation Details}

The reference implementation (Soccer-AI) is available at: [GitHub URL to be added]

Key components:
\begin{itemize}
    \item \texttt{fan\_enhancements.py}: Emotional dimension computation
    \item \texttt{rag.py}: Relational dimension (KG-RAG)
    \item \texttt{ai\_response.py}: 4D synthesis and generation
    \item \texttt{database.py}: Ground truth data access
\end{itemize}

\section{Glossary}

\begin{table}[h]
\centering
\begin{tabular}{ll}
\toprule
\textbf{Term} & \textbf{Definition} \\
\midrule
Embodied RAG & RAG with persona computed from data \\
Ground Truth & Real-world data determining emotion \\
Persona Trajectory & Path through 4D space over time \\
Dimensional Synthesis & Combining dimensions into state \\
Temporal Momentum & Predicted future persona state \\
\bottomrule
\end{tabular}
\end{table}

\end{document}
